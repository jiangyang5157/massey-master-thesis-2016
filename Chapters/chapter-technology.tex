%****************************************************************
% Chapter X
%****************************************************************
\chapter{Technology}
\label{chapter-technology}

As an immersive virtual reality application, there are few things need to consider. A device can be utilized and efficiently perform development tasks, including graphic system and network support; A markup language for decorating geographic data. There are some major technologies listed in this chapter.

%****************************************************************
\section{Virtual Reality Device}

There are following reasons for using Android smartphone as the virtual reality device. The intention is to identify immersive virtual reality device that not only low-cost but also a standard, customer-friendly device. That is the smartphone, and it had an incredibly fast growth trend in the last few years and a good promising market prospect \ref{fig:smartphone-shipments-forecast}. After all, it contains all the necessary sensors and positioning systems to measure motion and accurately track device movements - Six degrees of freedom (DOF) - position coordinates (x, y and z offsets) and orientation (yaw, pitch and roll angles). Additionally, 15\$ Google Cardboard kit turns Android or iOS smartphone to immersive virtual reality device. According to International Data Corporation (IDC), Android dominated the smartphone market with a share of 87.6\% in the worldwide \ref{fig:smartphone-os-market-share}. Moreover, there is an existing Google VR SDK \cite{google.vr-sdk.2016} for Android supports.

\begin{figure}[H]
\caption[Smartphone shipments forecase]{Smartphone shipments forecase \cite{tony.smartphone-market.2015}}
\label{fig:smartphone-shipments-forecast}
\centering
\includegraphics[width=0.7\textwidth, keepaspectratio]{Figures/smartphone-shipments-forecast.png}
\decoRule
\end{figure}

\begin{figure}[H]
\caption[Smartphone OS market share]{Smartphone OS market share \cite{idc.smartphone-os-market-share.2016}}
\label{fig:smartphone-os-market-share}
\centering
\includegraphics[width=0.7\textwidth, keepaspectratio]{Figures/smartphone-os-market-share.png}
\decoRule
\end{figure}

%****************************************************************
\section{OpenGL ES}

Android includes support for high-performance 2D and 3D graphics with the Open Graphics Library, specifically, the OpenGL ES API \cite{google.opengles.2016}. OpenGL ES is a branch of the OpenGL specification intended for embedded devices. The Google VR SDK requires the device has a minimum OpenGL ES 2.0 support. Table \ref{tab:opengles-spec-android} shows a version list of OpenGL ES API that Android supported.

\begin{table}[H]
\caption{OpenGL ES API specification supported by Android}
\label{tab:opengles-spec-android}
\centering
\begin{tabular}{l l l}
\toprule
\tabhead{OpenGL ES Version} & \tabhead{Android Version}\\
\midrule
OpenGL ES 1.0 & Android 1.0 and higher\\
OpenGL ES 1.1 & Android 1.0 and higher\\
OpenGL ES 2.0 & Android 2.2 (API level 8) and higher\\
OpenGL ES 3.0 & Android 4.3 (API level 18) and higher\\
OpenGL ES 3.1 & Android 5.0 (API level 21) and higher\\
OpenGL ES 3.2 (September 2016) & Android 7.0 (API level 24) and higher\\
\bottomrule
\end{tabular}
\end{table}

%****************************************************************
\section{Geographic Visualization Markup Language}

The Keyhole Markup Language (KML) can be combined with other supporting files such as imagery in a zip archive, producing a KMZ file. KML offers features for expressing geographic annotation and visualization. The annotations of KML features are not designed as machine-readable XML, but a human readable plain text or simple HTML. KML has Network links supported (\code{NetworkLink} is a KML facility of many \ref{fig:kml-schema}), which gives the power to serve content from a local or remote location. It generally used to distribute data to large numbers of clients. In another word, if the data requires an update, it has to be changed at the source location (remote server) and all users receive the updated data automatically.

As the markup language, more important than the satisfaction of needs is that the KML has been supported by many virtual globes and other GIS systems. It is becoming a de facto standard \cite{blower.sharing-visualizing.2007} that can be manipulated in other software if required.

\begin{figure}[H]
\caption[KML schema]{KML schema \cite{google.kml.2016}}
\label{fig:kml-schema}
\centering
\includegraphics[height=0.7\textheight, keepaspectratio]{Figures/kml-schema.png}
\decoRule
\end{figure}

%****************************************************************
\section{Network}
\label{section:network}

Real-time data are very important in the environmental sciences \cite{blower.sharing-visualizing.2007}. The key strengths of virtual reality applications are not only easy-to-use and intuitive nature, but also the ability to efficiently incorporate new data. Therefore, a web server is needed. A RESTful web server to support communication with the client, and a remote file server to synchronize data are included in the application.

Go (often referred to as golang \cite{google.golang.2016}) is an open source programming language, and it is compiled, concurrent, garbage-collected, statically typed language developed at Google in late 2007. It was conceived as an answer to some of the problems they were seeing and developing software infrastructure \cite{google.talk-golang.2012}. Surprisingly, the rise of Go was growing so fast that each month the contributors outside Go team itself are already more than the contributors inside the Go team. Additionally, Golang is well suited for developing RESTful API’s. Its \code{net/http} standard library provides a set of key methods for interacting via the HTTP protocol. For the above reasons, the Golang was selected for developing the server. 

On the client side (Android platform), Volley is being used for transmitting network data. It is an open sourced HTTP library that makes networking for Android apps easier and most importantly, faster \cite{google.volley.2016}. Also, application is taking use of Jsoup (Java HTML Parser \cite{joup.2016}) for analyzing HTML format response.

%****************************************************************
