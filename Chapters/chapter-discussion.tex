%****************************************************************
% Chapter X
%****************************************************************
\chapter{Discussion}
\label{chapter-discussion}

In this paper, I implemented an immersive virtual reality Android application includes a remote database server and a graphic display system for geographic data visualization. Popping a smartphone into the Google Cardboard container and then strapping it to people's head may sound like funny, but it works, and it is a super low-cost way ($15$ US dollars) to experience virtual reality compare to other particular virtual reality devices. After all, Android phone contains all the necessary sensors and positioning systems to track device movements which contribute to a development of complex virtual reality application. 

KML format data source has been used for decorating geographic data in the application. A KML format file not only provides basic needs for a decoration of geographic data but also has the ability to contain remote files by URL. It guarantees a real-time synchronous data which is important in the environmental sciences. It allows all or part of the dataset to be automatically refreshed by the URL, to ensure the user always sees the latest information. The human readable plain text is beneficial to publish and consume data in interoperable formats without the needs of technical assistance. Furthur more, KML has been adopted by more and more people and scientists across the different industry. Thus, there is a mass of geographic data resources created by KML have been shared within geographic communities. For instance, existing virtual globe software Google Map and Google Earth.

The application has an easy-to-use, intuitive nature working mode. On the one hand, a user can make a six degrees of freedom (DOF) movement. Although due to the sensor fusion creates a huge drift during the nasty double integration process, I alternatively using the Step sensor (pedometer) as the pedestrian navigation instead of Linear Acceleration sensor. It allows the user to move forward in the current heading that satisfies user for navigating through all scene. On the other hand, the application enables the ability for the user to intuitively interact with the scene for a better understanding of information. For example, select a \code{Placemark} and view the details of the \code{Placemark} on a popup message board; display a \code{Placemark} related OBJ model, or any further information from a URL, such as an image, summarized Wikipedia, or plain text.

There are five human senses provide the information and passed to our brain for capturing our attention: sight ($70$\%), hearing ($20$\%), smell ($5$\%), touch ($4$\%), and taste ($1$\%) \cite{mazuryk.vr.1996}. The immersive virtual reality certainly improved the feedback of sight sense and hearing. Although spatial sound is not included in the application yet, but by the given the existing Spatial Audio technology (such as \cite{google.spatial-audio.2016}), it can easily use the spatial audio as a simultaneous response for "fooling" the hearing sense.

There is a limitation of gesture recognition and perception technology, in particular with Android smartphones. As the modern phone, they are now being designed to be less physical key as possible. Therefore a user can only trigger a click (a touching) or focus on somewhere (staring at a target). On the other hand, PC-based pseudo-3D virtual globes are manipulated by the mouse and keyboard which allows more straightforward interactive actions.

The application guarantees $60$ FPS when the there is less than $250$ \code{Placemark} exist in the scene. Although this number is not satisfied as a complex geographic visualization application, it has huge potential for releasing the restricted performance issues. As we can see from the performance testing result \ref{chapter-performance}, application it has a smooth memory usage, and the optimized ray-model intersection and model updates are extremely successful. Once an future optimization of \code{glDrawElements} is implemented, there will be no pressure for thousands of objects existing at the same time.

The application satisfy basic needs as an immersive virtual reality application for geographic data visulization, along with a remote file server and RESTful API support. However, it still has many unfinished optimizations and features due to a limited time. As we can see from the KML schema \ref{fig:kml-schema}, application only supports a small part of KML facility.

One of the important features in geographic data visualization is Level Of Detail (LOD) render based on the distance from the eyes to the target area. Detail textures could be separately prepared, and attached as the circumstances may require. It can also provide a solution to visualize a large amount of overlapping data (enable layers on distance). Additionally, along with more geometric shapes support in the application (such as lines and polygons), LOD allows the user to see the details not only geographical map but also the architectural structure on different floors.

A key requirement in environmental scientists is to be able to visualize four-dimensional data (i.e. time-dependent three-dimensional data). It does not just visualize the environment data from different data files which created from the different period of time, or a fake real-time data visualization by refreshing in a certain frequently rate. But, the ability to visualize dynamic graphic animation on a flexible timeline. It improves user understanding of environmental data to a higher level. In other words, an animation transform from one piece of time-dependent data to another, just like watching the 3D movie.

After all, in the immersive virtual reality visualization point of view, there is always room for improvement. When it related to human intuitive nature system, the user experience will always be not real enough compare to the real world.

%****************************************************************
