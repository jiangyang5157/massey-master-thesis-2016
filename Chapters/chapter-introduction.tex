%****************************************************************
% Chapter X
%****************************************************************
\label{chapter-introduction}
\chapter{Introduction}

Undoubtedly VR has attracted a lot of interest of people in last few years, along with a mess of VR related products, such as Google Cardboard that be able to turn Smartphone to VR device, 3D 360 camera, and 3D VR headset are developed by different manufacturers. 

Three-dimensional (3D) representations of objects and spaces are frequently used to improve human understanding \parencite{tuttle.virtual-globes.2008}. Therefore VR is appropriate to be used as geographic data visualization.  

%****************************************************************
\section{Background and Overview}

VR provides an easy, powerful, intuitive way of human-computer interaction. The user can watch and manipulate the simulated environment in the same way we act in the real world, without any need to learn how the complicated user interface works. Therefore many applications like flight simulators, architectural walkthrough or data visualization systems were developed relatively fast \parencite{mazuryk.virtual-reality.1996}.

%****************************************************************
\section{Literature Review and Limitations}

****************************************************************\\%####
What is known?\\
What is unknown?\\
Identifying Limitations\\

%****************************************************************
\section{Aims and Objectives}

Due to the equipment costs and lacks of VR softwares, by compare with successful virturl globes, geographic data visualization with VR has not yet as popular as virturl globe. 

Our Objectives is to explore and implement a VR project that runs on a minimum equipment costs VR device. By doing the project, we develop a geographic data visualization VR software that includes both front and back end.

The project is expected to show geographic data visualization with VR can be easily used as virturl globe, and it will be widespread in the future.

%****************************************************************
