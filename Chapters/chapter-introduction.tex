%****************************************************************
% Chapter X
%****************************************************************
\label{chapter-introduction}
\chapter{Introduction}

There has been an increased interest in the exploration of Virtual Environments (VE) \cite{huang.java-cgi-vr.2002}, sometimes called Virtual Reality. The first fifteen years of the 21st century has seen significant, rapid advancement in the development of virtual reality became much more dynamic, the term Virtual Reality itself became extremely popular, and there was a broad range of applications were developed relatively fast. They offer significant benefits in many areas, such as architectural walkthrough, scientific visualization, modeling, designing and planning, training and education, telepresence and teleoperating, cooperative working and entertainment \cite{mazuryk.vr.1996}. Among these applications, virtual reality technology has been proved it offers new and exciting opportunities for users to interact visually with and explore 3D geographic data \cite{huang.java-cgi-vr.2002}. 

%****************************************************************
\section{Overview and Objectives}

In the previous practices of visualizing geographic data with virtual reality were mostly using 3D representations of objects and displayed them on a 2D monitor. This pseudo-3D nature of virtual reality is not enough for offering what people desire, and they want able to step into the world and interact with it, instead of watching the 2D projection image on the monitor. That is the ultimate motivation of virtual reality technology - a real 3D experience with immersive stereoscopic 3D visuals.

Nowadays, given the rapid advancement in the development of computer technology especially small and powerful mobile technologies have exploded while prices are continually driven down. The rise of smartphones with high-density displays and 3D graphics capabilities has enabled a generation of lightweight and functional virtual reality devices. It seems clear that we step into a critical period of immersive virtual reality industry while multiple virtual reality related products that finally seem to enter the market constantly. However, still, there is a lack of research in both theory and practice way for visualizing geographic data in the immersive virtual reality. 

In order to evaluate how geographic data visualization with immersive virtual reality affect user interfaces and human-computer interactions, an immersive virtual reality application that composed of a database management system and a graphic display system for geographic data visualization is developed for the purpose of this study. This paper also highlighted the essential considerations that get involved in such implementations: the ranges and capabilities of any necessary sensors to create the immersive virtual reality; evaluation of the minimum equipment costs; shared geographic visualization markup language; geodetic-mapping coordinates; performance of 3D graphic. In this thesis, a background of geographic data visualization is presented. Then, details of the related technology and implementation are described. Finally, there is a discussion and conclusion around the results and future research.

%****************************************************************
\section{Background}
\label{section:background}

The Geographic Information System (GIS) is a broad term, it often refers to many different technologies, processes, and methods that designed to capture, store, manipulate, analyze, manage, and present spatial or geographic data \cite{wiki.gis.2016}. A GIS combines a database management system and a graphic display system that tie to the process of spatial analysis \cite{rhyne.virtual.1997}. Indeed, GIS has been widely used in the analysis of the Earth and environmental data, mostly used in 2D, map-based systems. However, significant problems have had exposed. First, GIS itself only handle 2D data; second, displays are limited to spatial views of the data; third, the capability of supporting user interaction with negligible data \cite{rhyne.visualization-gis.1994}. Nevertheless, the concept of taking advantage of GIS to visualize the earth and environmental sciences data has been already studied for a long time in both theory and practice area, and that is called Virtual Globe (VG) technology. Although, most of the virtual globe products are pseudo-3D nature based, but still, they allow users to interact with an environment that makes the data and information present easier to understand \cite{tuttle.virtual-globes.2008}. Therefore, it dramatically has become a powerful tool for navigating geospatial data in 3D and contribute to all kind of communities across different usage till now. 

Essentially, the success of virtual globes is the improvement of human understanding in the following aspect. \cite{tuttle.virtual-globes.2008}. 

\begin{description}
	\setlength{\parskip}{0pt}
    \item[$\bullet$ pseudo-3D] Allows users to interact with an environment that they naturally understand.
	\item[$\bullet$ Transportability] Digital data are easily transported.
	\item[$\bullet$ Scalability] Can be view at any scale.
	\item[$\bullet$ Interactivity] Provides an interactive experience for users.
	\item[$\bullet$ Choice of topics] Topics can be changed dynamically, and presented individually or together.
	\item[$\bullet$ Currency] The data presented can be of any age, including real time.
	\item[$\bullet$ Client-side] Puts the power in the hands of the user.
\end{description}

Virtual globe technology is beneficial to education. For teaching spatial thinking, virtual globes offer tremendous opportunities, and it can be expected that they will greatly influence how a new generation will perceive space and geographic processes, said by Nuernberger \cite{nuernberger.vr-classroom.2006}. It also helps scientific collaboration research, such as the EarthSLOT \cite{earthslot.2016}. Moreover, Butler points out virtual globes can be used as an invaluable tool in disaster response \cite{butler.vg.2006, nourbakhsh.mapping-disaster-zones.2006}. Virtual globe technology has many exciting possibilities for environmental science. The easy-to-use, intuitive nature system, provide attractive and efficient means and methods for simultaneously visualizing four-dimensional environmental data from different sources that driving a greater understanding and user experience of the Earth system \cite{blower.sharing-visualizing.2007}.

The Open Geospatial Consortium (OGC) is committed to making quality open standards for the global geospatial group. These standards were decided through a consensus based process and are freely available for anyone to sharing of the world's geospatial data. They have made contributions to many communities including government, commercial organizations, non-governmental organizations, academic and research organizations \cite{ogc.2016}. To use a markup language maintained by OGC for the creation of 3D geographic maps and associated spatial data allows scientists to publish the latest information in a single, simple data file format without technical assistance. More importantly, it potentially allows environmental scientists to visualize 4D data (i.e. time-dependent three-dimensional data) from data files created in the different period.

A markup language maintained by the Open Geospatial Consortium \cite{ogc.2016} plays an essential role in virtual reality implementation. By taking the use of a markup language, scientists are able to publish data in a single, simple data file format without technical assistance. In spite of capabilities vary from products to products, but virtual globes always provide support for a file format data exchange and the ability to simultaneously display multiple datasets. Blower et al. point out \cite{blower.sharing-visualizing.2007} Google Earth which has the largest community creates Keyhole Markup Language (KML) \cite{google.kml.2016} files as its primary method for visualizing data (KML is an international standard maintained by the OGC); NASA World Wind \cite{nasa.world-wind.2016} imports data from tile servers, OGC web services and limited support for KML, it has more focus toward scientific users; ArcGIS Explorer \cite{esri.arcgis-explorer.2016} is a lightweight client to the ArcGIS Server, it can import data in a very wide range of GIS formats, including KML. Some of the virtual globes products are using Virtual Reality Modeling Language (VRML) \cite{wiki.vrml.2016} that is a language for describing 3D objects and interactive scenes on the World-Wide Web (WWW) \cite{wiki.www.2016}, It has been superseded by X3D \cite{wiki.x3d.2016}.

The KML is a somewhat limited language. It can only describe simple geometric shapes, such as points, lines, and polygons, and is not extensible. By compared with Geography Markup Language (GML), in many respects, GML 3.0+ is much more sophisticated and allows the rich description of geospatial features such as weather fronts and radiosonde profiles. For the above reasons, KML is currently not suitable as a fully-featured, general-purpose environmental data exchange format. Nonetheless, it still earns the acceptance from an increasing number of scientists. From the point of view of usability, KML spans a gap between very simple (e.g. GeoRSS) and more complex (GML) formats, which makes it easy for non-technical scientists to share and visualize simple geospatial information which can then be manipulated in other applications if required. After all, it is important to be aware of that virtual geographic data visualization (or KML) does not attempt to replace more sophisticated systems.

In recent years, given the rapid development of technique progress in computers and pipelined 3D graphics, the immersive virtual reality not only frequent occurrences nearly in all sorts of media, but also it has a mess of related products developed by manufacturers over the world. For example, Google has released similar virtual reality products such as the Google Cardboard, a DIY immersive virtual reality headset that drives by smartphone; Samsung has taken this concept further with products such as the Galaxy Gear, which is mass produced and contains features such as gesture control; the 3D camera that can capture a 360 degrees field of view. However, it is not mature enough to eliminate the equipment limitation and becomes a universal technology in daily human life by comparison to the pseudo-3D virtual reality technology. For instance, when it comes to exploring, routing or getting to places, most people should just reach for Google Earth or Google Map.

Immersive virtual reality provides an easy used, powerful, intuitive way of user interaction. The user can experience and manipulate the simulated 3D environment in the same way they act in the real world, without any preparation or understanding of the complicated user interface works. It soon became a perfect tool that is beneficial to architects, designers, physicists, chemists, doctors, surgeons, etc. Without a doubt VR has a great potential to change our life, the expectation from this technology is much more than it can offer yet \cite{mazuryk.vr.1996}.

%****************************************************************
