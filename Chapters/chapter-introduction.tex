%****************************************************************
% Chapter X
%****************************************************************
\label{chapter-introduction}
\chapter{Introduction}

%****************************************************************
\section{Background and Overview}

There has been an increased interest in the exploration of Virtual Environments (VE) \parencite{huang.java-cgi-vr.2002}, sometimes called Virtual Reality (VR). Since beginning of 1990s when the development in the area of virtual reality became much more dynamic and the term Virtual Reality itself became extremely popular, a wide range of applications were developed relatively fast, which offers great benefits in many area, such as "architectural walkthrough", "scientific visualization", "modeling, designing and planning", "training and education", "telepresence and teleoperating", "cooperative working" and "entertainment" \parencite{mazuryk.vr.1996}. Among these applications, virtual reality technology was proved that it offers new and exciting opportunities for users to interact visually with and explore 3D geo-data \parencite{huang.java-cgi-vr.2002}.

In the past, GIS were mostly 2D, map-based systems, but the concept of taking advantage of GIS to visualize the earth and environmental sciences data has been already studied for a long time. That is called Virtual Globe (VG) technology, most of the virtual globe products use 3D representations of objects and display them onto a 2D monitor. This pseudo-3D nature of virtual globes allows users to interact in an environment that makes the data and information present easier to understand \parencite{tuttle.virtual-globes.2008}. Then it has become a powerful tool for navigating geospatial data in 3D and contribute to all kind of communities across different usage till now. 

Nowadays, with the premise of pipelined 3D graphics and efficient algorithms for visualization, GIS can be used either with a workstation window-based interface or with an immersive virtual reality environment \parencite{koller.virtual-gis.1995}. In spite of the immersive virtual reality frequent occurrences nearly in all sort of media, along with a mess of related products, such as Google Cardboard that be able to turn a Smartphone to VR device, 3D 360 camera, and different kinds of VR headset are developed by manufacturers over the world, but it is not mature enough to eliminate the equipment limitation and becomes universal by comparison to virtual globe. Also, there is a lack of research on implementation of geographic data visualization with immersive virtual reality device. Therefore, we cannot yet say whether or not immersive virtual reality for geographic data visualization is better than other visualization and analysis approaches for certain data, if so, by how much; what basic considerations would be involved in geographic data visualization with immersive virtual reality.

%****************************************************************
\section{Literature Review}
\label{section:literature-review}

\parencite{mazuryk.vr.1996} present a historical overview of virtual reality. Given the current rapid development of virtual GIS technology, they made a point of the motivation of virtual reality technology is that people always want more, they want able to step into the world and interact with it, instead of watching the 2D projection image on the monitor. VR provides an easy used, powerful, intuitive way of user interaction. The user can experience and manipulate the simulated 3D environment in the same way they act in the real world, without any preparation or understanding of the complicated user interface works. It soon became a perfect tool that is beneficial to architects, designers, physicists, chemists, doctors, surgeons etc. Without a doubt VR has a great potential to change our life, the expectation from this technology is much more than it can offer yet. \parencite{mazuryk.vr.1996} also discussed on an interesting idea that they came up with: new invention brings fear, the more potential it has, the bigger the danger can be.

Visualization and GIS methodologies are often used to examine the Earth and environmental sciences data. A GIS combines a database management system and a graphic display system that tie to the process of spatial analysis \parencite{rhyne.visualization-gis.1994, rhyne.virtual.1997}. Indeed, GIS is now widely used in the analysis of environmental data, but there are significant problems: first, GIS itself only handle 2D data; second, displays are generally limited to spatial views of the data; third, the capability of supporting user interaction with  negligible data \parencite{rhyne.visualization-gis.1994}. They also point out that map is not only a tool for people getting from here to there but a way of organizing knowledge to make it understandable.

The success of virtual globes \parencite{tuttle.virtual-globes.2008} is not only because the improvement of human understanding from its pseudo-3D representations of objects and spaces, but also the five features: transportability (digital data are easily transported), scalability, interactivity (users are in control of the experience), choice of topics (topics can be combined or presented individually or ), currency (ability to adjust the data available to any given time period), and client-side \parencite{tuttle.virtual-globes.2008}. Virtual globes can be beneficial to education ('For teaching spatial thinking, Virtual Globes offer tremendous opportunities, and it can be expected that they will greatly influence how a new generation will perceive space and geographical processes.' \parencite{nuernberger.vr-classroom.2006}), scientific collaboration research (such as the EarthSLOT \parencite{earthslot.2016}), and disaster response (VG is an invaluable tool in disaster response \parencite{butler.vg.2006, nourbakhsh.mapping-disaster-zones.2006}). Virtual globe technology has many exciting possibilities for environmental science. The easy-to-use, intuitive nature system, provide attractive and effective means and methods for simultaneously visualizing four-dimensional environmental data from different sources that driving a greater understanding and user experience of the Earth system \parencite{blower.sharing-visualizing.2007}. 

A markup language maintained by the Open Geospatial Consortium \parencite{ogc.2016} plays an essential role in virtual reality implementation. By taking the use of a markup language, scientists are able to publish data in a single, simple data file format without technical assistance \parencite{blower.sharing-visualizing.2007}. In spite of capabilities vary from products to products, but virtual globes always provide a support for a file format data exchange and the ability to simultaneously display multiple datasets. \parencite{blower.sharing-visualizing.2007} points out Google Earth which has the largest community creates Keyhole Markup Language (KML) \parencite{Google.kml.2016} files as its primary method for visualizing data (KML is an international standard maintained by the OGC); NASA World Wind \parencite{nasa.world-wind.2016} imports data from tile servers, OGC web services and a limited support for KML, it has more focus toward scientific users; ArcGIS Explorer \parencite{esri.arcgis-explorer.2016} is a lightweight client to the ArcGIS Server, it can import data in a very wide range of GIS formats, including KML. Some of the virtual globes products are using Virtual Reality Modeling Language (VRML) \parencite{wiki.vrml.2016} that is a language for describing 3D objects and interactive scenes on the World-Wide Web (WWW) \parencite{wiki.www.2016}, It has been superseded by X3D \parencite{wiki.x3d.2016}.

%****************************************************************
\section{Aims and Objectives}

Our Objectives is to explore how geographic data visualization with immersive virtual reality affect user interfaces and human-computer interactions, 3D computer graphics for virtual reality programing, and the ranges and capabilities of any necessary sensors. Our approach to achieving this is to firstly identify an appropriate VR devices, that is to say, a common, customer-friendly devices that require a minimum equipment cost. After that, we develop an immersive virtual reality based geographic data visualization application that takes advantage of GIS and the implementation includes both front (client) and back-end (web server).

%****************************************************************
