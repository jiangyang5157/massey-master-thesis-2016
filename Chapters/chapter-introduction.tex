%****************************************************************
% Chapter X
%****************************************************************
\label{chapter-introduction}
\chapter{Introduction}

Three-dimensional (3D) representations of objects and spaces are frequently used to improve human understanding \parencite{tuttle.virtual-globes.2008}. Therefore VR is appropriate to be used as geographic data visualization.  However, due to the equipment costs and lacks of VR softwares, by compare with successful virturl globes, geographic data visualization with VR has not yet as popular as virturl globe. 

%****************************************************************
\section{Background and Overview}

Undoubtedly VR has attracted a lot of interest of people in last few years, along with a mess of VR related products, such as Google Cardboard that be able to turn Smartphone to VR device, 3D 360 camera, and 3D VR headset are developed by different manufacturers. 

%****************************************************************
\section{Literature Review and Limitations}

****************************************************************\\%####
What is known?\\
What is unknown?\\
Identifying Limitations\\

****************************************************************\\%####
VR provides an easy, powerful, intuitive way of human-computer interaction. The user can watch and manipulate the simulated environment in the same way we act in the real world, without any need to learn how the complicated user interface works. Therefore many applications like flight simulators, architectural walkthrough or data visualization systems were developed relatively fast \parencite{mazuryk.virtual-reality.1996}.


****************************************************************\\%####
\parencite{mazuryk.virtual-reality.1996}\\
Three-dimensional objects have six degrees of freedom (DOF): position coordinates (x, y and z offsets) and orientation (yaw, pitch and roll angles).\\

\parencite{blower.sharing-visualizing.2007}\\
We explain how we have used Web Services to connect virtual globes with diverse data sources and enable more sophisticated usage such as data analysis and collaborative visualization.\\
Recently there have been great advances in the development of “virtual globes”: software applications that display a three-dimensional representation of the entire Earth, usually based on satellite imagery, upon which new information can be superimposed.\\
(Near) real-time data can be displayed in Google Earth using the NetworkLink facility in KML. This facility allows all or part of a KML dataset to be refreshed at a fixed rate or based on the expiration time given in the HTTP header. A user can therefore download a fixed KML file containing the NetworkLink and Google Earth will automatically refresh the link, ensuring that the user always sees the latest information.\\
A key requirement for environmental scientists is to be able to visualize four dimensional data (i.e. time-dependent three-dimensional data).\\

\parencite{tuttle.virtual-globes.2008}\\
Three-dimensional (3D) representations of objects and spaces are frequently used to improve human understanding.\\
The pseudo-3D nature of virtual globes allows people to interact in an environment that they naturally comprehend and that makes the data and information presented easier to understand.\\
Transportability relates to the fact that virtual globes are based on digital data and, therefore, can be transported as easily as any other digital information.\\
Scalability is a powerful aspect of virtual globes.\\
Interactivity is a crucial part of the power of virtual globes. Users are in control of the experience and can adjust the view, scale, data layers, and more.\\
Choice of topics is an important difference from traditional representations of geospatial data. Virtual globes have the power of combining multiple topics in one media.\\
Currency is the ability to easily adjust the data available to any given time period. The data can be static or have an update time ranging from years to real time.\\
The rapid development of virtual globes and their supporting infrastructure of imagery and servers represent a convergence of mainstream information technology and geography. Geography is increasingly becoming a central organizing principal for managing, analyzing, and visualizing the world’s information.\\

%****************************************************************
\section{Aims and Objectives}

Our Objectives is to explore and implement a VR project that runs on a minimum equipment costs VR device. By doing the project, we develop a geographic data visualization VR software that includes both front and back end.

The project is expected to show geographic data visualization with VR can be easily used as virturl globe, and it will be widespread in the future.

%****************************************************************
