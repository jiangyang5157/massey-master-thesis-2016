%****************************************************************
% Chapter X
%****************************************************************
\label{chapter-conclusion}
\chapter{Conclusion}

%****************************************************************
The immersive virtual reality provides a highly integrated easy-to-use, intuitive system. It's efficient and simultaneous property provides an attractive way for 3D geographic data visualization. By comparison to virtual globes, geographic data visualization with immersive virtual reality device allows to do similar things, but, due to the limitation of human-machine interaction, gesture recognition, and perception, the VR is not yet able to do everything that the pseudo-3D virtual globes can do. For instance, virtual globes are manipulated by the mouse. However, it has the potential to do more than people expected in the future.

Android SDK for virtual reality development is well supported. By making use of Google Cardboard and Android smartphone is not only the most convenient way to experience immersive virtual reality but also easy to spread the virtual reality product. 

The most logical way to calculate the movement in an immersive virtual reality environment is to first use Gyroscope to measures angular velocity relevant to the body, or in other words, to get the device orientation. Then, using Accelerometer to inject the correction term that keeps the orientation correct on gravity, a correction due to the magnetic north from Compasses is also required. However, it is hard to get an accurate position out of them, due to a horrible drift comes from the nasty double integration process. Alternatively, a pedestrian navigation was implemented in the application based on the Step sensor and a certain algorithm for velocity.

KML is not only a human-readable markup language can and very suit for visualizing geographic data, but also it has very well compatibility with current major virtual globes, such as the well-known Google Earth. Moreover it also powerful enough to describe a small subregion, such as a building hierarchical plan. On the other hand, immersive virtual reality also can be used as a tool that able to visualizing different sort of data, or natural system by integrating another or new spatial markup language.

%****************************************************************
