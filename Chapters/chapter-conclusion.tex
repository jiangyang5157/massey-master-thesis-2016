%****************************************************************
% Chapter X
%****************************************************************
\label{chapter-conclusion}
\chapter{Conclusion}

%****************************************************************
The immersive virtual reality provides a highly integrated easy-to-use, intuitive real-time 3D GIS for geographic data visualization. Due to the limitation of human-machine interaction, the VR is not yet able to do everything that the pseudo-3D virtual globes can do, but it has the potential to do more than people expected when there is a revolution for gesture recognition and perception.

There are at least three reasons indicate that Android Phones are extremely suitable to use as an immersive virtual reality device. First, always about the money, 15\$ Google Cardboard kit turns Android or iOS smartphone to immersive virtual reality device; second, the existing VR specific open source SDK provided by Google, includes necessary graphic, and spatial audio development; third, Android includes support for high-performance graphics with OpenGL.

The most logical way to calculate the movement in the immersive virtual reality environment is to use Gyroscope to measures angular velocity relative to the body, or in other words, to get the device orientation.
Then, using Accelerometer to inject the correction term that keeps the orientation correct with respect to gravity, and a correction due to the magnetic north from Compasses is also required. However, it is really hard to get an accurate position out of them due to a horrible drift comes from the nasty double integration process. Alternatively, a pedestrian navigation was implemented for this project is based on the Step sensor (pedometer) with a certain algorithm to calculate the velocity was turn out working very well. The limitation of this approach is can only move forward in the current heading direction.

KML is not only a human-readable markup language can and very suit for visualizing geographic data, but also it has very well compatibility with current major virtual globes, such as the well-known Google Earth. Moreover it also powerful enough to describe a small subregion, such as a building hierarchical plan. On the other hand, immersive virtual reality also can be used as a tool that able to visualizing different sort of data, or natural system by integrating another or new spatial markup language.

%****************************************************************
